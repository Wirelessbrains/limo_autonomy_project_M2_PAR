\newpage
\section{Otmizacao e trajetoria }


Apos um visualizaçao em que um camera se move em uma trajetoria que se 
paraece com incosaedro, durante a visualizaçao foi garantido que a todo
momento a camera ve pelo menos duas tags;

A primeira etapa a realizar é fazer mapaemento do nosso ambiente. Sendo 
assim, definir a primeira tag referencia o calculadas tranformaçoes 
relativas para as tags subsequentes. dado que eu sei onde todas as tags 
estao localizada posiçao e orientaçao, eu posso localizar o meu 
robo no ambiente.

A problemantica é que o robo nao consegue ver todas as tags ao mesmo tempo 
e também, dado que o robo ve as tags em sequencia pequenos erros de 
detecçao da posicao do robo sao acumulados a medida em que o robo se
move

Logo o mapaemento deve ser em caideia fazendo as transformada de referencial
a partir de uma tag referencia, ou a tag mundo.

A camera é colocada como pose de referencia e recebe a pose relativa da
tagA, primeira tag vista, ao mesmo tempo, é possivel ver a tagB, Assim
atraves destas poses é possivel cacula a pose da tag B em relaçao a tagA:

\begin{equation}
    T_{A \to B} = T_{Cam \to A}^{-1} * T_{Cam \to B}
\end{equation}

\begin{itemize}
    \item $T_{A \to B} = (R_A, \bm{t}_A)$: Pose da Tag B em relaçao a A.
    \item $T_{Cam \to A} = (R_A, \bm{t}_A)$: Pose da Tag A vista pela Câmera.
    \item $T_{Cam \to B} = (R_B, \bm{t}_B)$: Pose da Tag B vista pela Câmera.
    \item $R$: Matriz de rotaçao
    \item $t$: Vetor de posiçao
\end{itemize}

A tranformaçao de inveresa é caculada da seguinte forma:

\begin{equation}
    T^{-1} = (R^T, -R^T \bm{t})
\end{equation}

dada que o as orientaçoes das tags sao dadas em quartenion, é feita a conversao 
para matrizes de rotacao:


\begin{equation}
R = \begin{bmatrix}
q_0^2 + q_1^2 - q_2^2 - q_3^2 & 2(q_1q_2 - q_0q_3) & 2(q_0q_2 + q_1q_3) \\
2(q_1q_2 + q_0q_3) & q_0^2 - q_1^2 + q_2^2 - q_3^2 & 2(q_2q_3 - q_0q_1) \\
2(q_1q_3 - q_0q_2) & 2(q_0q_1 + q_2q_3) & q_0^2 - q_1^2 - q_2^2 + q_3^2
\end{bmatrix}
\end{equation}

Assim é possivel calcular a rotaçao relativa($R_{rel}$):

\begin{equation}                                                          
R_{rel} = R_A^T R_B                                                   
\end{equation}

Dado que a detecao da tag também nos fornece a posiçao, assim também
é possivel calcular: a translaçao relativa ($\bm{t}_{rel}$):

\begin{equation}                                                          
     \bm{t}_{rel} = R_A^T (\bm{t}_B - \bm{t}_A)                            
\end{equation}

Assim

\begin{equation}
    T_{A \to B} = (R_{rel}, \bm{t}_{rel})
\end{equation}

também é possivel realisar o calculo atravez de matrizes de transformação homogenea:

                                                          
                                                             
                   
\begin{equation}
\begin{bmatrix} 
R_{rel} & \bm{t}_{rel} \\ 
\mathbf{0}_{1 \times 3} & 1 
\end{bmatrix} = 
\begin{bmatrix} 
R_A & \bm{t}_A \\ 
\mathbf{0}_{1 \times 3} & 1 
\end{bmatrix}^{-1} 
\begin{bmatrix} 
R_B & \bm{t}_B \\ 
\mathbf{0}_{1 \times 3} & 1 
\end{bmatrix}
\end{equation}

Assim, dado que o nosso ambiente é conhecido, ou seja a pose das todas as trafomaçoes 
das tags sao conhecidas, de igual modo é possivel localizar a pose do robo.

A partir da detecçao da pose da tag em realçao a camera é feito uma tranformaçao
inversa da pose da tag até a camera para encontrar qual a pose da comaera em ralaçao a tag

\begin{equation}
    T_{W,C} = T_{W,T} * (T_{C,T})^{-1}
\end{equation}

Em que: 

\begin{equation}
    $T_{W,T}$ é a posição da Tag no seu mapa (onde a Tag 0 é a origem).
    $T_{C,T}$ é a detecção (a Tag vista pela Câmera).
    $(T_{C,T})^{-1}$ é a pose da Câmera em relação à Tag.
\end{equation}

Se a tag que a camera vé nao é a tag zero, mas dado que o ambiente foi mapeado, sabemos a pose
desta tag em relaçao a tag0 e asse por meio de tranformaçao inversa é possivel calcular a pose 
da camera.

Para pose da camera quanto mais tags forem vista mais precisao temos para a pose da camera,
visto que podemos calcular a media da pose da camera no mundo em ralaçao a cada tag. Por 
um metodo chamado SLERP, que envolve a converçao da rotaçao para quartenion. 

\subsection{Fusão de Poses da Câmera via Interpolação Esférica (SLERP)}

Quando a câmera observa simultaneamente múltiplas etiquetas, as estimativas individuais da pose da câmera no referencial global ($\mathbf{P}_{W,C}$) são fundidas para reduzir ruídos de detecção. O processo de fusão é dividido em três etapas: conversão de representação, média de translação e interpolação esférica de rotação.

\subsubsection{Conversão de Matriz de Rotação para Quaternion}
As matrizes de rotação $\mathbf{R} \in SO(3)$ obtidas do sistema de visão são convertidas para quaternions unitários $\mathbf{q} = [w, x, y, z]^T$. Dada uma matriz $\mathbf{R}$ com elementos $r_{ij}$, o componente escalar $w$ é calculado a partir do traço da matriz ($Tr(\mathbf{R})$):
\begin{equation}
    w = \frac{1}{2} \sqrt{1 + r_{11} + r_{22} + r_{33}}
\end{equation}
Os demais componentes $(x, y, z)$ são derivados das diferenças entre os elementos simétricos da matriz:
\begin{equation}
    x = \frac{r_{32} - r_{23}}{4w}, \quad y = \frac{r_{13} - r_{31}}{4w}, \quad z = \frac{r_{21} - r_{12}}{4w}
\end{equation}

\subsubsection{Média da Translação (Posição)}
A posição final da câmera é calculada através da média aritmética simples das estimativas de posição $\mathbf{p}_1$ e $\mathbf{p}_2$:
\begin{equation}
    \mathbf{p}_{avg} = \frac{\mathbf{p}_1 + \mathbf{p}_2}{2}
\end{equation}

\subsubsection{Interpolação Esférica Linear (SLERP)}
Para a rotação, utiliza-se o \textit{Spherical Linear Interpolation} (SLERP), que realiza a interpolação ao longo do arco da hiperesfera unitária. Para duas rotações $\mathbf{q}_1$ e $\mathbf{q}_2$ com fator de peso $t=0.5$:

\begin{enumerate}
    \item \textbf{Caminho Curto:} Calcula-se o produto escalar $\cos(\theta) = \mathbf{q}_1 \cdot \mathbf{q}_2$. Se $\cos(\theta) < 0$, inverte-se o sinal de um dos quaternions ($\mathbf{q}_2 = -\mathbf{q}_2$) para garantir o caminho mais curto.
    \item \textbf{Cálculo do SLERP:}
    \begin{equation}
        \mathbf{q}_{avg} = \frac{\sin((1-t)\theta)}{\sin(\theta)}\mathbf{q}_1 + \frac{\sin(t\theta)}{\sin(\theta)}\mathbf{q}_2
    \end{equation}
\end{enumerate}